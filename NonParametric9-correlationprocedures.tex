Relationships between variables. To express a relationship between two variables one usually computes the correlation coefficient. Nonparametric equivalents to the standard correlation coefficient are Spearman R, Kendall Tau, and coefficient Gamma (see Nonparametric correlations). If the two variables of interest are categorical in nature (e.g., "passed" vs. "failed" by "male" vs. "female") appropriate nonparametric statistics for testing the relationship between the two variables are the Chi-square test, the Phi coefficient, and the Fisher exact test. In addition, a simultaneous test for relationships between multiple cases is available: Kendall coefficient of concordance. This test is often used for expressing inter-rater agreement among independent judges who are rating (ranking) the same stimuli.

Nonparametric Correlations

The following are three types of commonly used nonparametric correlation coefficients (Spearman R, Kendall Tau, and Gamma coefficients). Note that the chi-square statistic computed for two-way frequency tables, also provides a careful measure of a relation between the two (tabulated) variables, and unlike the correlation measures listed below, it can be used for variables that are measured on a simple nominal scale.

Spearman R. Spearman R (Siegel & Castellan, 1988) assumes that the variables under consideration were measured on at least an ordinal (rank order) scale, that is, that the individual observations can be ranked into two ordered series. Spearman R can be thought of as the regular Pearson product moment correlation coefficient, that is, in terms of proportion of variability accounted for, except that Spearman R is computed from ranks.

Kendall tau. Kendall tau is equivalent to Spearman R with regard to the underlying assumptions. It is also comparable in terms of its statistical power. However, Spearman R and Kendall tau are usually not identical in magnitude because their underlying logic as well as their computational formulas are very different. Siegel and Castellan (1988) express the relationship of the two measures in terms of the inequality: More importantly, Kendall tau and Spearman R imply different interpretations: Spearman R can be thought of as the regular Pearson product moment correlation coefficient, that is, in terms of proportion of variability accounted for, except that Spearman R is computed from ranks. Kendall tau, on the other hand, represents a probability, that is, it is the difference between the probability that in the observed data the two variables are in the same order versus the probability that the two variables are in different orders.

-1 £ 3 * Kendall tau - 2 * Spearman R £ 1

Gamma. The Gamma statistic (Siegel & Castellan, 1988) is preferable to Spearman R or Kendall tau when the data contain many tied observations. In terms of the underlying assumptions, Gamma is equivalent to Spearman R or Kendall tau; in terms of its interpretation and computation it is more similar to Kendall tau than Spearman R. In short, Gamma is also a probability; specifically, it is computed as the difference between the probability that the rank ordering of the two variables agree minus the probability that they disagree, divided by 1 minus the probability of ties. Thus, Gamma is basically equivalent to Kendall tau, except that ties are explicitly taken into account.
