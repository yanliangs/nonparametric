ONE SAMPLE: THE WILCOXON TEST
Just as is the case for the sign test, the Wilcoxon test can be used to test a null hypothesis concerning the
value of the population median. Because the Wilcoxon test considers the magnitude of the difference between
each sample value and the hypothesized value of the median, it is a more sensitive test than the sign test. On the
other hand, because differences are determined, the values must be at least at the interval scale. No assumptions
are required about the form of the population distribution.
The null and alternative hypotheses are formulated with respect to the population median as either a one-
sided or two-sided test. The difference between each observed value and the hypothesized value of the median
is determined, and this difference, with arithmetic sign, is designated d:d ¼ (X  Med 0). If any difference is
equal to zero, the associated observation is dropped from the analysis and the effective sample size is thereby
reduced. The absolute values of the differences are then ranked from lowest to highest, with the rank of 1
assigned to the smallest absolute difference. When absolute differences are equal, the mean rank is assigned to
the tied values. Finally, the sum of the ranks is obtained separately for the positive and negative differences. The
smaller of these two sums is the Wilcoxon T statistic for a two-sided test. In the case of a one-sided test, the
smaller sum must be associated with the directionality of the null hypothesis. Appendix 10 identifies the critical
values of T according to sample size and level of significance. For rejection of the null hypothesis, the obtained
value of T must be smaller than the critical value given in the table.
