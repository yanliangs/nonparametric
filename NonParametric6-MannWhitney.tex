Mann-Whitney-Wilcoxon Test

Two data samples are independent if they come from distinct populations and the samples do not affect each other. Using the Mann-Whitney-Wilcoxon Test, we can decide whether the population distributions are identical without assuming them to follow the normal distribution.

Example
In the data frame column mpg of the data set mtcars, there are gas mileage data of various 1974 U.S. automobiles.

> mtcars$mpg 
 [1] 21.0 21.0 22.8 21.4 18.7 ...
Meanwhile, another data column in mtcars, named am, indicates the transmission type of the automobile model (0 = automatic, 1 = manual). In other words, it is the differentiating factor of the transmission type.

> mtcars$am 
 [1] 1 1 1 0 0 0 0 0 ...
In particular, the gas mileage data for manual and automatic transmissions are independent.

Problem
Without assuming the data to have normal distribution, decide at .05 significance level if the gas mileage data of manual and automatic transmissions in mtcars have identical data distribution.

Solution
The null hypothesis is that the gas mileage data of manual and automatic transmissions are identical populations. To test the hypothesis, we apply the wilcox.test function to compare the independent samples. As the p-value turns out to be 0.001817, and is less than the .05 significance level, we reject the null hypothesis.

> wilcox.test(mpg ~ am, data=mtcars) 
 
        Wilcoxon rank sum test with continuity correction 
 
data:  mpg by am 
W = 42, p-value = 0.001871 
alternative hypothesis: true location shift is not equal to 0 
 
Warning message: 
In wilcox.test.default(x = c(21.4, 18.7, 18.1, 14.3, 24.4, 22.8,  : 
  cannot compute exact p-value with ties
Answer
At .05 significance level, we conclude that the gas mileage data of manual and automatic transmissions in mtcar are nonidentical populations.


%====================================================================================== %
\subsection*{ONE SAMPLE: THE WILCOXON TEST}
\begin{itemize}
\item Just as is the case for the sign test, the Wilcoxon test can be used to test a null hypothesis concerning the
value of the population median. 
\item Because the Wilcoxon test considers the magnitude of the difference between
each sample value and the hypothesized value of the median, it is a more sensitive test than the sign test. On the
other hand, because differences are determined, the values must be at least at the interval scale. 
\item No assumptions
are required about the form of the population distribution.
\end{itemize}

%====================================================================================== %
The null and alternative hypotheses are formulated with respect to the population median as either a one-
sided or two-sided test. The difference between each observed value and the hypothesized value of the median
is determined, and this difference, with arithmetic sign, is designated d:d ¼ (X  Med 0). If any difference is
equal to zero, the associated observation is dropped from the analysis and the effective sample size is thereby
reduced. The absolute values of the differences are then ranked from lowest to highest, with the rank of 1
assigned to the smallest absolute difference. When absolute differences are equal, the mean rank is assigned to
the tied values. 

%====================================================================================== %
Finally, the sum of the ranks is obtained separately for the positive and negative differences. The
smaller of these two sums is the Wilcoxon T statistic for a two-sided test. In the case of a one-sided test, the
smaller sum must be associated with the directionality of the null hypothesis. Appendix 10 identifies the critical
values of T according to sample size and level of significance. For rejection of the null hypothesis, the obtained
value of T must be smaller than the critical value given in the table.

%====================================================================================== %
\end{document}
