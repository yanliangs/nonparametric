Cochran's Q test
https://en.wikipedia.org/wiki/Cochran%27s_Q_test

In statistics, in the analysis of two-way randomized block designs where the response variable can take only two possible outcomes (coded as 0 and 1), Cochran's Q test is a non-parametric statistical test to verify whether k treatments have identical effects.[1][2][3] It is named for William Gemmell Cochran. Cochran's Q test should not be confused with Cochran's C test, which is a variance outlier test. Put in less technical terms, requires that there only be a binary response (success/failure or 1/0) and that there be 2 or more matched groups (groups of the same size). The test assesses whether the proportion of successes is the same between groups. Often used to assess if different observers of the same phenomenon have consistent results amongst themselves (interobserver variability).

%--------------------------------------%
Background[edit]
Cochran's Q test assumes that there are k > 2 experimental treatments and that the observations are arranged in b blocks; that is,

Treatment 1	Treatment 2	\cdots	Treatment k
Block 1	X11	X12	\cdots	X1k
Block 2	X21	X22	\cdots	X2k
Block 3	X31	X32	\cdots	X3k
\vdots
\vdots
\vdots
\ddots
\vdots
Block b	Xb1	Xb2	\cdots	Xbk
Description[edit]
Cochran's Q test is

H0: The treatments are equally effective.
Ha: There is a difference in effectiveness among treatments.
The Cochran's Q test statistic is


T = k\left(k-1\right)\frac{\sum\limits_{j=1}^k \left(X_{\bullet j} - \frac{N}{k}\right)^2}{\sum\limits_{i=1}^b X_{i\bullet}\left(k-X_{i\bullet}\right)}
where

k is the number of treatments
X• j is the column total for the jth treatment
b is the number of blocks
Xi • is the row total for the ith block
N is the grand total
Critical region[edit]
For significance level α, the critical region is


T > \chi^2_{1-\alpha,k-1}
where Χ21 − α,k − 1 is the (1 − α)-quantile of the chi-squared distribution with k − 1 degrees of freedom. The null hypothesis is rejected if the test statistic is in the critical region. If the Cochran test rejects the null hypothesis of equally effective treatments, pairwise multiple comparisons can be made by applying Cochran's Q test on the two treatments of interest..

Assumptions[edit]
Cochran's Q test is based on the following assumptions:

A large sample approximation; in particular, it assumes that b is "large".
The blocks were randomly selected from the population of all possible blocks.
The outcomes of the treatments can be coded as binary responses (i.e., a "0" or "1") in a way that is common to all treatments within each block.
Related tests[edit]
When using this kind of design for a response that is not binary but rather ordinal or continuous, one instead uses the Friedman test or Durbin tests.
The case where there are exactly two treatments is equivalent to McNemar's test, which is itself equivalent to a two-tailed sign test.
